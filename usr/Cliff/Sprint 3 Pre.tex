\documentclass[slidestop,compress,mathserif,c]{beamer}
\usepackage{setspace}
\usepackage{color}
\usepackage{hyperref}

\usetheme{Antibes}
\usecolortheme{lily}

\title{3-D Modeling based on Drone Photometry}
\subtitle{Sprint 3 - Presentation \& Demonstration}
\author{Chunpeng Wang \and Xinggao Yang \and Ayush Shirsat \and Julie Park}
\institute{\emph{Electrical \& Comuputer Engineering Department, Boston University \\ Mechanical Engineering Department, Boston University}}
\date{\today}

\makeatletter
% add a macro that saves its argument
\newcommand{\footlineextra}[1]{\gdef\insertfootlineextra{#1}}
\newbox\footlineextrabox
 
% add a beamer template that sets the saved argument in a box.
% The * means that the beamer font and color "footline extra" are automatically added. 
\defbeamertemplate*{footline extra}{default}{
  \begin{beamercolorbox}[ht=2.25ex,dp=1ex,leftskip=\Gm@lmargin]{footline extra}
    \insertfootlineextra
    % \par\vspace{2.5pt}
  \end{beamercolorbox}
}
 
\addtobeamertemplate{footline}{%
  % set the box with the extra footline material but make it add no vertical space
  \setbox\footlineextrabox=\vbox{\usebeamertemplate*{footline extra}}
  \vskip -\ht\footlineextrabox
  \vskip -\dp\footlineextrabox
  \box\footlineextrabox%
}
{}
 
% patch \begin{frame} to reset the footline extra material
\let\beamer@original@frame=\frame
\def\frame{\gdef\insertfootlineextra{}\beamer@original@frame}
\footlineextra{}
\makeatother


\setbeamertemplate{footline}{%
  \leavevmode%
  \hbox{%
    \begin{beamercolorbox}[wd=.25\paperwidth,ht=2.25ex,dp=1ex,center]{author in head/foot}%
      \usebeamerfont{author in head/foot}{Group 9}
    \end{beamercolorbox}%
    \begin{beamercolorbox}[wd=.5\paperwidth,ht=2.25ex,dp=1ex,center]{title in head/foot}%
      \usebeamerfont{title in head/foot}\insertshorttitle
    \end{beamercolorbox}%
    \begin{beamercolorbox}[wd=.25\paperwidth,ht=2.25ex,dp=1ex,right]{date in head/foot}%
      \usebeamerfont{date in head/foot}\insertshortdate{}\hspace*{2em}
      \insertframenumber{} / \inserttotalframenumber\hspace*{2ex}
    \end{beamercolorbox}}%
  \vskip0pt%

}

\begin{document}

	\begin{frame}
		\titlepage
	\end{frame}

    \begin{frame}{GitHub \& Trello}
        \begin{itemize}
            \item GitHub: \\
            \href{https://github.com/3DMBDP/3D_Drone_Reconstruction}{https://github.com/3DMBDP/3D\_Drone\_Reconstruction}
            \item Trello: \\
            \href{https://trello.com/b/wNXNliLy/sprint-3}{https://trello.com/b/wNXNliLy/sprint-3}
        \end{itemize}
    \end{frame}

	\begin{frame}{Outline}
		\tableofcontents
	\end{frame}

    \section{Previous Work}
    \begin{frame}{Previous Work}
        Images from three different angles:
        \begin{enumerate}
            \item Down-Angle
            \item Horizontal-Angle
            \item \alert{Up-Angle}
        \end{enumerate}
    \end{frame}
    
    \section{Progress in Sprint 3}
    \subsection{Image Subtraction}
    \begin{frame}{Progress in Sprint 3}{Image Subtraction}
        \begin{itemize}
            \item 20 Down-Angle images.
            \item Subtracted from the original images.
            \item SIFT to do the feature extraction.
        \end{itemize}
    \end{frame}
    
    \subsection{Feature Detector - BRISK}
    \begin{frame}{Progress in Sprint 3}{Feature Detector - BRISK}
    Feature Detectors:
        \begin{itemize}
            \item SIFT
                \begin{itemize}
                    \item Lowe's binary
                    \item OpenSIFT
                \end{itemize}
            \item \alert{BRISK}: The output does not match, needs transformation
        \end{itemize}
    \end{frame}
    
    \section{Result of Sprint 3}
    \begin{frame}{Result of Sprint 3}
        \begin{itemize}
            \item 76 Down-Angle pictures (subtracted)
            \item Result in two communication towers.
        \end{itemize}
    \end{frame}
    
    \section{Future Work}
    \begin{frame}{Future Work}
        \begin{itemize}
            \item Bad quality images:\\
            Sunlight towards the camera $\rightarrow$ glare in the picture.
            \item Most of the feature points are on background.
        \end{itemize}
    \end{frame}
    
    \section*{}
		\begin{frame}
			\begin{center}
				\Huge{Q \& A}
			\end{center}
		\end{frame}
    
    \section*{}
		\begin{frame}
			\begin{center}
				\Huge{Thank You!}
			\end{center}
		\end{frame}

\end{document}
